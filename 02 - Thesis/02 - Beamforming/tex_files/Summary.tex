\chapter*{Abstract}
\setheader{Abstract}

This thesis presents a comparison between two different beamforming algorithms, the delay-and-sum beamformer (DSB) and the minimum variance distortionless response (MVDR) beamformer using both objective and subjective metrics. Simulations and measurements have been performed on near-field scenarios with an ad-hoc microphone array. Microphones from multiple smartphones are used to record the audio. Smartphone microphones are designed to have a distinct directivty gain pattern. In this research, the MVDR algorithm is extended to compensate for these directivity responses. In simulated scenarios, without the directivities of the microphones, the average improvement in segmental signal to noise ratio is $5.3 dB$ for DSB and $11.9 dB$ for MVDR. In all the simulated scenarios, the MVDR beamforming algorithm outperforms the DSB algorithm. Results from measurements in an office room scenario only showed an improvement when applying the DSB algorithm. The results from measurements in the anechoic chamber at the TU Delft show $1.9 dB$ improvement for DSB and $2.8 dB$ for MVDR.
Compensating for the directivity in the implemented MVDR mainly added noise to the signal.