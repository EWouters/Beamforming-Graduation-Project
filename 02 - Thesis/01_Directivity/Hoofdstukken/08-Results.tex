\section{Results}
\label{sec:results}
After measuring, processing and equalizing, some of the results are presented at the end of this chapter (from page \pageref{fig:res_NX506_pluis} to \pageref{fig:res_NX501_pluis}).
For there are conducted a lot of measurements, only a handful of result is displayed in this thesis for comparison.
For more results, the writers can be contacted.

The results that will be discussed are from four different measurement set-ups:
\begin{itemize}
\item \nexus, labelled with number six in mid-air, for a full sphere (Figure \ref{fig:res_NX506_pluis});
\item \nexus, labelled with number six face up on the surface, for a sphere with $\phi\in[0,135]^\circ$ (Figure \ref{fig:res_NX506_FU});
\item \nexus, labelled with number six face down on the surface, for a sphere with $\phi\in[0,135]^\circ$ (Figure \ref{fig:res_NX506_FD}); and,
\item \nexus, labelled with number one in mid-air, due to time constrains only measured for a sphere with $\phi\in[0,90]^\circ$ (Figure \ref{fig:res_NX501_pluis}).
\end{itemize}

There are a few things to notice from these given results.
When the smartphone microphone is turned away from the loudspeaker, the gain is less then when the smartphone microphone is turned toward the loudspeaker (Figure \ref{fig:res_NX506_pluis_90}).
For elevations higher or smaller than $90^\circ$, this effect is less (Figure \ref{fig:res_NX506_pluis_42}, \ref{fig:res_NX506_pluis_138}).

%Effect of surface
The addition of the surfaces changes the reception of the signals, especially the signals from below the surface (Figure \ref{fig:res_NX506_FU_135}, \ref{fig:res_NX506_FD_135}).

%Difference between two phones
Between the two phones there is a visible difference between the behaviour of the directivity (Figure \ref{fig:res_NX501_pluis}), the gain of the the phone labelled with number one seems higher than the gain of the phone labelled with number six.
In the next section, these results will be placed in different perspectives.