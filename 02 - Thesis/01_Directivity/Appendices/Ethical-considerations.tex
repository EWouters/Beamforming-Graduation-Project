\chapter{Ethical considerations}
\label{app:ethical}
\setheader{Ethical considerations}

This chapter in the thesis is not technically part of the thesis.
It describes some ethical considerations related to (smartphone) microphone arrays and their applications.
This chapter is included in all thesis works of the project group, e.g. also the work of Van Wijngaarden and Wouters \cite{BAP:ErikNiels} and Bosma and Smeding \cite{BAP:RoySjoerd}. This chapter contains \emph{opinions} of the authors and is not a technical document.

\section{The need for ethical discussion}
In creating the beamforming system, we envisioned a single purpose: speech enhancement in ad-hoc teleconference calls. Unfortunately, history teaches us some well-intended inventions turn out to be dangerous to the point where regulation is needed to ban it. Examples are numerous: tetraethyllead was added to gasoline starting in the 1920s but phased out worldwide from the 1970s because of its environmental impact. Radithor is another 1920s example; a ``wonder medicine'' in its heyday, it was subsequently discovered to be extremely toxic to the point that the main marketeer for the drugs' ``jaw fell off''.
Although these examples are very extreme, it is worthwhile examining the potential influence on society of our system-to-be.

\section{Unintentional use cases}
The question we must ask ourselves thus becomes: what are some possible unforeseen use cases of our product? The possibility of a surveillance state (or company) using ad-hoc beamforming techniques to snoop on all its residents springs to mind. The revelations of NSA whistle blower Edward Snowden have documented just how far some American and European states are willing to go to keep tabs on their citizens. Moreover, they have revealed great levels of cooperation from tech-industrial giants such as Google, Facebook, Apple and others.

\section{Beamformer contribution}
But how much would a smartphone beamforming system contribute to any level of mass-surveillance? The truth is that beamformers only work when the locations of smartphones are known to extreme accuracy. Such accuracy is not achievable with sensors fitted on even the latest smartphones. Furthermore, indoor localization of smartphones has thus far always relied on either large, extra hardware (RF beacons) or a sonic beacon. Neither seem appropriate to a state wanting to covertly listen in on conversations.

\section{Applications}
Our prime focus has been with companies deploying the technology in teleconferencing situations. We do not seek out to be the next Skype, Hangouts or appear.in. A company using such software must be fully aware of the consequences of privacy invasion or potential information leakage caused by such software. Our solution would not be connected directly to the internet; it could, for example, serve as a virtual microphone input to the computer. What is done with the recorded audio is completely up to the user. 

\section{Final remarks}
Theoretically, if the technology becomes more mature, it may become a weapon in the arsenal of surveillance states. But the technology is not remotely there yet. Ad-hoc beamforming on smartphones is still relatively uncharted territory and the lack of silent, unnoticable synchronization options kills off a potential to listen in discretely.
We honestly see no potential for abuse of the technology investigated in our research.

