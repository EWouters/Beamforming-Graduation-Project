
% This LaTeX was auto-generated from MATLAB code.
% To make changes, update the MATLAB code and republish this document.

\documentclass{article}
\usepackage{graphicx}
\usepackage{color}

\sloppy
\definecolor{lightgray}{gray}{0.5}
\setlength{\parindent}{0pt}

\begin{document}

    
    
\subsection*{Contents}

\begin{itemize}
\setlength{\itemsep}{-1ex}
   \item Rec2file Class
   \item Properties
   \item Methods
\end{itemize}
\begin{verbatim}
classdef rec2file
\end{verbatim}


\subsection*{Rec2file Class}


\begin{verbatim}Class to record audio samples to file as .mat or .wav (or others)\end{verbatim}
    
\begin{verbatim}Example:
 obj = rec2file
 obj.save([1,2,3,4],1)\end{verbatim}
    

\subsection*{Properties}

\begin{verbatim}
    properties
        fileName                                        % Name of the audio file to save including extention
        filePath                                        % Path of the audio file to save
        Title = sprintf('Recorded on %s',datestr(now)); % Set Title to current date and time
        Comment = 'Beamforming Audio recording';        % Add comment to audio file
    end
\end{verbatim}


\subsection*{Methods}

\begin{verbatim}
    methods
        function save(obj,samples,fs)
            if isempty([obj.fileName obj.filePath])
                [obj.fileName,obj.filePath] = uiputfile(...
                    {'*.wav','Waveform Audio File Format (*.wav)';...
                    '*.mat','MAT-files (*.mat)';...
                    '*.*',  'All Files (*.*)'},...
                    'Save File','fileName');
            end
            if isempty(strfind(obj.fileName,'.mat'))
                audiowrite(fullfile(obj.filePath,obj.fileName),samples,fs);
            else
                audio.samples = samples;
                audio.fs = fs; %#ok<STRNU>
                save(fullfile(obj.filePath,obj.fileName),'audio');
            end
        end
    end
\end{verbatim}
\begin{verbatim}
end
\end{verbatim}



\end{document}
    
