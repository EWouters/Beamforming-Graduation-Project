\documentclass[a4paper]{article}
\usepackage[a4paper,top=1.25in,right=1in,bottom=1.25in,left=1in]{geometry}
\usepackage[utf8]{inputenc} 
\usepackage{amsmath} %some standard math functions
\usepackage{amssymb} %more mathematical symbols
\usepackage{amsbsy} %enables bold mathematics
\usepackage{amsthm} %enable theorem statements
\usepackage[small, bf, hang]{caption} %captions for floats (images etc.)
\usepackage{subcaption} %subcaptions for floats (images etc.)
\usepackage{enumitem}
\usepackage{adjustbox}
\usepackage{lastpage}
%\usepackage{mathtools}
\usepackage{fancyhdr}
\usepackage{titlesec}
\usepackage{setspace}
\usepackage{graphicx}
\usepackage{float}
\usepackage{placeins}
\usepackage{subcaption}
\usepackage{url}
\usepackage{color}
\usepackage{sectsty}
\usepackage{microtype}
\pagestyle{fancy}
\setlength{\headwidth}{\textwidth}
\lhead{} % leave left header empty
\chead{\textsf{Acoustic Enhancement via Beamforming Using Smartphones}}
\rhead{} % leave right header empty
\lfoot{} % leave left footer empty
\cfoot{} % leave center footer empty
\rfoot{\textsf{Page~\thepage~of \pageref{LastPage}}}
\renewcommand{\headrulewidth}{0.3pt}
\renewcommand{\footrulewidth}{0pt}
\newcommand{\norm}[1]{\left\lVert#1\right\rVert}
\renewcommand \thesection{\arabic{section}} %
\renewcommand{\thesubsection}{\arabic{section}.\arabic{subsection}} % Remove from subsections
\renewcommand{\thesubsubsection}{\arabic{section}.\arabic{subsection}.\arabic{subsubsection}}	
% Nice enumerations
\newlist{enum}{enumerate}{1}
\setlist[enum]{label=\textbf{[\arabic*]}} % \arabic or \alpha
\setlist{itemsep=-5pt}
\definecolor{title-gray}{gray}{0.45} %grijstint voor headers
\allsectionsfont{\sffamily\color{title-gray}} %sans-serif in headers
\newcommand{\matlab}{\textsc{Matlab} } %fancy Matlab commando
\onehalfspacing %ruime tekst marges
%\usepackage[colorlinks=false]{hyperref} %klikbare links in PDF

\begin{document}
\title{{\sffamily Schedule of Requirements for Bachelor Graduation Project\\Acoustic Enhancement via Beamforming Using Smartphones}}
\author{{\sffamily\small S. Bosma, R. Smeding}}
\date{}
\maketitle
\section{Introduction}
As part of the greater acoustic enhancement system, the communication between smartphones and the central computer is essential. The term \emph{system} used below refers to the complete system of Android phones, central computer and wireless router.

\section{Requirements concerning the intended use}
\textbf{Requirements concerning the Android application.}
\begin{enumerate}
\item The Android application must have an easy-to-use graphical interface.
\item The Android application must connect over Wi-Fi to the dedicated wireless router handling communication between said application and the central computer.
\item The Android application must stream, in real-time, the data acquired from its microphone after receiving the corresponding signal from the central computer.
\item The Android application must stop streaming the recorded sound after receiving a stop signal.
\item The Android application must adhere strictly to a defined communication protocol with the central computer.
\end{enumerate}

\textbf{Requirements concerning the \matlab application.}
\begin{enumerate}
\item The \matlab program must connect over Wi-Fi to the dedicated wireless router handling communication between said application and the central computer.
\item The \matlab program must be capable of handling multiple phone connections simultaneously.
\item The \matlab program must adhere strictly to a defined communication protocol with the phones.
\item The \matlab program must provide the beamforming algorithms with discrete, synchronized `blocks' with predefined size of sampled audio data from all phones.
\item The \matlab program must also provide to the beamforming algorithms a unique phone identifier along with said audio data.
\end{enumerate}
\section{Requirements from an ecological point of view}
\begin{enumerate}
\item A few minutes after disconnection, the Android application must be shut down automatically to prevent unnecessary battery drain.
\item After a few minutes of discontinued use, the \matlab application should not prevent the computer from going into a state of low-power usage (e.g. hibernation).
\end{enumerate}

\section{Requirements concerning the system}
\begin{enumerate}
\item The system must be usable in a typical office environment such as meeting rooms.
\item The system must not exhibit noticeable latency in it's operation.
\end{enumerate}

\subsection{Usage characteristics} 
\begin{enumerate}
\item The Android application must run on LG Nexus 5 mobile phones.
\item The computer program must run on computers running Windows 7 with networking features.
\end{enumerate}

\subsection{Production and commissioning characteristics}
\begin{enumerate}
\item \matlab Compiler Runtime is needed to use the developed software, this software is available for free from Mathworks. 
\end{enumerate} 
\subsection{Recycling features}

\section{Requirements regarding the production}
\begin{enumerate}
\item The Android application will be developed using the Android SDK available from Google for free. The used version will be release 24.2.
\item The \matlab code will be developed using \matlab version R2014b. 
\item Java applications included will be developed using JDK version 1.7.0.
\end{enumerate}

\section{Requirements concerning the recycling system}
\begin{enumerate}
\item Both the Android application and the resulting computer application must be easily removable from the phone, respectively the central computer.
\end{enumerate}

\section{Requirements from a strategic, marketing and sales point of view}
\begin{enumerate}
\item Misschien iets over distributie?
\end{enumerate}
%Please do not change!
\newpage
\bibliography{references/references}{}
\label{sec:bibliography}
\bibliographystyle{references/IEEEtran}
\end{document}

\end{document}
